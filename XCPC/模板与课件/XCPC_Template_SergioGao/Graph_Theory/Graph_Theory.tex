\chapter{图论}

主要记录一些比较常用的高级算法。

\section{Tarjan 相关图论算法}

叫 Tarjan 的图论算法其实有很多。所以 Tarjan 实际上可以说是一类思想。
它可以说是洞察了 dfs 树的性质,考虑了返祖边,跨越边(对于无向图,实际上并不可能存在)等。

\subsection{求强连通分量}

这是对于有向图,求出若干极大两两可互相到达的点集。

一些没有写上去但可能会用到的 trick:

对于强连通分量,存在容易算法判断是否存在非零环。
(提示:如果存在两个权值不一样的环,那么一定存在非零环。)

\lstinputlisting{Graph_Theory/tarjan/tarjan_scc.cpp}



\section{网络流}

网络流在 XCPC 中常考的两类:

第一类是最大流或者最小割。这类一般建完图用 Dinic 算法解决即可。

特别地,最经典的一类题型:等价于求二分图的最大匹配。
而最大匹配用最大流来跑又快又好,并且还能顺便证明二分图最大匹配 = 最小点覆盖。

二分图中:最大匹配(最大流) = 最小点覆盖(最小割) = 点数 - 最大独立集

第二类是最小费用最大流。这类一般建完图用 Dij 费用流解决即可。
大多数费用流算法都和其流 $f$ 相关。构造时一般是那最大流作为可以控制的约束。
特别地,最经典的一类题型:等价于求二分图的最大权匹配。

\subsection{Dinic 最大流}

一般的时间复杂度会估计到 $O(n^2 m)$,但很多题目的建图有特殊性,跑不满。
特别的,如果拿来处理二分图最大匹配,复杂度可以估到 $O(n^3)$。

\lstinputlisting{Graph_Theory/flow/dinic_flow.cpp}

\subsection{Dijkstra 费用流}

首先,费用流一般是要自己构造的,所以一般有意义的问题不会有真正的负环。
然而负权是允许的。通过调整正负,可以把最小费用流的算法改写成最大费用流。

Dijkstra 费用流也就是在保证初始能赋出一个合理的势能函数的前提下,
用 Dijkstra 来找到从源点到汇点的最短路。

最暴力的做法是初始做一遍 spfa,这个复杂度可能会达到 $O(n m)$。
或者根据题目的特殊性。
有一些问题它给出时就已经是让我们求如何最大化一些势能差之和。

除去这个初始化,复杂度是 $O(f m \log m)$。
要注意,像是二部图最大权匹配,$f$ 取决于较小的那一部,所以不要误认为其等于总点数。

\lstinputlisting{Graph_Theory/flow/dijkstra_flow.cpp}

