\section{Bingo 模型}

\begin{frame}
  \frametitle{Bingo 模型描述}

  给定一个 $n\times m$ 的棋盘。
  
  说到棋盘,自然可以定义其行/列。
  由于它们有类似的结构,我们定义 strip 来统称这两类元素。
  
  strips 之间允许有交。

  通过等概率过程让一些数填充到这个棋盘上。

  假设一些 strips 构成集合 $STRIPS$。
  包含所有 strips 的记作 $STRIPS_0$。

  对于一种填充方案 $\{b_{nm}\}$,定义最小 bingo 数为:
  $$
  \min_{I\in STRIPS_0}\{\max_{i \in I}\{ b_i\}\}
  $$

  求所有填充方案最小 bingo 数的总和。(对于无限过程,则改为求期望)。
  
  把内部套的那层看作 strip 到实数的映射 $G(I):=\max_{i \in I}\{ b_i\}$, 则由 min-max 容斥,明显下面的式子成立

\end{frame}

\begin{frame}

  $$
  \min_{I\in STRIPS_0}\{G(I)\} = 
  \sum_{
    \emptyset\not= STRIPS \subseteq STRIPS_0  
  }
  (-1)^{|STRIPS|-1}
  \max_{I\in STRIPS}\{G(I)\}
  $$  

  考虑这个求和式的局部
  $$
  \max_{I\in STRIPS}\{G(I)\} = \max_{I\in STRIPS}\{\max_{i \in I}\{ b_i\}\}
  $$

  很容易发现,他就是 STRIPS 内 strips 交出来的这些元素,
  求当前填充方案下这些元素的最大值。
  
  由于填充过程是等概率的,这个值往往只和去重后、具体含了多少个元素相关。
  对于 $r$ 行 $c$ 列,里面显然有 $r\cdot m + c\cdot n - r\cdot c$ 个元素。(这也算一个简单的容斥)
  
  所以我们预处理出 $F(\text{cnt}), \text{cnt} = 1,2,\cdots, n\cdot m$,然后暴力累加
  $$
  \sum_{r+c\ge 1, r=0,1,\cdots,n, c = 0,1,\cdots, m}
  (-1)^{r+c-1}\binom{n}{r}\binom{m}{c} F(r\cdot m + c\cdot n - r\cdot c)
  $$

\end{frame}

\begin{frame}
  \frametitle{【2024ICPC-Nanjing】I. Bingo}

  \begin{block}{题目描述(简化)}
  给定两个整数 $n$、$m$,以及一个长度为 $n\cdot m$ 的整数序列 $a_1, a_2, \cdots, a_{nm}$,我们要用这些整数填充一个 $n$ 行 $m$ 列的网格。

  对于一种填充方案 $\{b_{nm}\}$,定义最小 bingo 数为:

  $$
  \min_{\text{strip }I}\{\max_{i \in I}\{ b_i\}\}
  $$

  请计算所有该序列排列(共 $(nm)!$ 种)下最小 bingo 数的总和。由于答案可能很大,请输出结果对 $998\,244\,353$ 取模。
  \end{block}

\end{frame}

\begin{frame}
  \newcommand{\cnt}{\text{cnt}}

  这题要预处理的是,选中长度为 $n\cdot m$ 的整数序列中的 $\cnt$ 个元素后,遍历所有排列,
  这些元素最大值的累加。
  
  容易想到这样的做法:不妨把输入的 $\{a\}$ 排序,从头到尾扫一遍,目前扫到 $k$,
  数一下有多少种排列这个元素是最右边的。$\binom{k-1}{\cnt-1}$。
  可以看出 $k$ 从 $\cnt$ 遍历到 $n\cdot m$。

  请注意 $n\cdot m$ 是常数,后面我们会发现它的锚点作用。
  
  总之

  $$
  F(\cnt) = \sum_{k = \cnt }^{n\cdot m} \binom{k-1}{\cnt - 1} a_{k}
  $$

  简单拆一下组合数,把与 $k$ 无关的因子提出来

  $$
  \frac{F(\cnt)}{\cnt!} = \sum_{k = \cnt }^{n\cdot m} (k-\cnt)! \times \left( (k-1)!a_{k} \right)
  $$  

  这个能高效求吗?

\end{frame}

\begin{frame}
  \newcommand{\cnt}{\text{cnt}}

  $$
  \sum_{k = \cnt }^{n\cdot m} (k-\cnt)! \times \left( (k-1)!a_{k} \right)
  $$

  我特意区分了简单的数值乘法 $\cdot$ 和特殊意味的乘法 $\times$。

  看左边,是从 $0$ 开始增到 $n\cdot m - \cnt$。这里 $0$ 是一个锚点。

  看右边,是从 $\cnt$ 开始增到 $n\cdot m$。这里 $n\cdot m$ 是一个锚点。

  what if... 把右边 reverse 成从 $0$ 开始增到 $n\cdot m - \cnt$?

  原本同步的 $k$,右边 reverse 后,就变成一个增 1 一个减 1 了。

  枚举下标总和固定,想到了什么?这完全就是两个数组的卷积的某一位!最好能先想象出卷积过程的图形,再符号化成公式

  $$
  ( \{ N! \}* \{ c_{N} \} )_{\cnt} = \sum\limits_{j=0}^{n\cdot m -cnt} j! \times c_{\cnt-j}
  $$

  只需要定义 $c_{n\cdot m - j} = (j-1)! \cdot a_j  $ 就等价了。除了提取操作,不包含 $cnt$,所以 uniformly 用 NTT 预处理一下。
  

\end{frame}

\begin{frame}
  \frametitle{【杭电春季联赛25】宾果游戏}
  
  \begin{block}{题目描述}  
    一个 $n \times m$ 的表格,玩 bingo 游戏。

    每轮随机报出一个位置,在表格中给这个位置做上标记,
    当某一行/一列上的格子全部被标记时,游戏结束。

    表格中有 $k$ 个位置被墨水污染无法做上标记。

    请注意,主持人依然会报出被墨水污染的位置,也就是说,每个位置被报到的概率都还是 $\dfrac{1}{n \times m}$。

    请输出游戏结束的期望轮数对 998244353 取模的结果,若游戏无法结束,则输出-1。

  \end{block}

  这题属于是训练性质的缝合题,我们只列出关键 trick。剩余内容可以从其它题中找到。

\end{frame}

\begin{frame}

  \begin{exampleblock}{题意转化}
    给每个位置赋予一个最早被标记的时间。
    
    某个 strip $I$ 全被标记的最早时间 $G(I)$,也就是其包含的元素的最大值。
    某个 strip $I$ 全被标记就结束了,所以就是 $\min\{G(I)\}$。
    
    所以还是在求最小 bingo 数。

  \end{exampleblock}  

  此外还需要一个模型

  \begin{exampleblock}{部分抽奖问题 Partial Coupon Collector Problem}
    对于 $N$ 个等概率发生的事件,单位时间只能发生一个,如果选中 $k$ 个,那么这 $k$ 个全部发生过(也就是最晚的那个发生)的时间期望是 
    
    $$
    \frac{1}{k/N} + \frac{1}{(k-1)/N} + \cdots+\frac{1}{1/N}
    $$

    这可以简单地通过考虑几何分布得到(即期望等于概率的倒数)

  \end{exampleblock}

\end{frame}
