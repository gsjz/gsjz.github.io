\subsection{李超线段树}

\subsubsection{值域 $10^6$,针对线段}

要求在平面直角坐标系下维护两个操作:

\begin{itemize}
    \item 在平面上加入一条线段。记第 $i$ 条被插入的线段的标号为 $i$。
    \item 给定一个数 $k$,询问与直线 $x = k$ 相交的线段中,交点纵坐标最大的线段的编号。
\end{itemize}

\lstinputlisting{Convex_Optimization/Convex_Datastructure/lichao_tree1.cpp}

这个模板其实可能不够好,也不常用。主要是它的时间复杂度会和定义域大小相关。有些问题要查 1e9 就寄了。

\subsubsection{值域 $10^9$,针对直线}

更常用的应该是没有定义域限制,功能一样,然后拿来优化一些 dp。

下面这个模板展示了原来的 dp 和用李超线段树优化后的用法。

\href{https://codeforces.com/contest/2107/problem/F2}{CF2107F2}

\lstinputlisting{Convex_Optimization/Convex_Datastructure/lichao_tree2.cpp}

有的人会误认为内存要开成 $N\log V$,其中 $V$ 为 $10^9$,而大多数情况下存一个直线是需要至少一个 int 一个 long long 的,如果 $N$ 是 $10^6$ 再乘个 30 就很容易炸。实际上这是一种误解。

仔细观察实现过程,这个线段树没有分割后分别递归,而是只会往一侧走。

每次存下直线后都会直接停止,从而可以知道每次 add 操作只会新建一个结点,所以内存开成 $N$ 即可。

另外还有一个经典问题

Q:这种问题可能是什么原因,孩子调半天没搞明白

\begin{itemize}
    
\item 想用 vector 动态开点,每次用 .size() 来得到当前的 id(本地测试能出正确答案,OJ 上 RE)

\item 也是 vector,但是定义的时候预先定义大小,另外开一个 int 类型来记录当前 id(本地测出来和 1 一样,OJ 上 AC)

\end{itemize}

A:递归中存在对 vector 的引用,结果中途给 vector 扩容了,导致引用失效。

\subsubsection{若干个树共享内存池}

有些题目得不同组分别维护李超树,也不太能事先知道每棵加进的直线数,只保证加进的直线数总和,前面又说了目前这个代码不适合做 vector 扩容,
那么目前只能共享一下内存池。由于我第一次知道 Cpp 的静态变量和 Java 不一样,类内声明完后还得在外面声明一次,所以目前的模板还有点臃肿。

需要用到这个写法的题目其它部分太长,只贴一个不需要这个写法的简单题的应用 

\lstinputlisting{Convex_Optimization/convex_datastructure/lichao_tree3.cpp}

